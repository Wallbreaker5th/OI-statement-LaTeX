\documentclass{statement}
\usepackage{tabularx}
\usepackage{makecell}


\title{NOI 2021 模拟赛}
\subtitle{NOI 2021 Simulation}
\author{破壁人五号}
\date{\today}
\begin{document}
    \begin{titlingpage}
        \maketitle
        \begin{center}
        \begin{tabularx}{\textwidth}{|X|X|X|X|}
        \hline
            题目名称 & 句号路径 & 表达式插值 & 哥德巴赫\\
        \hline
            目录 & \texttt{cracket} & \texttt{fxpr} & \texttt{gib}\\
        \hline
            可执行文件名 & \texttt{cracket} & \texttt{fxpr} & \texttt{gib}\\
        \hline
            输入文件名 & \texttt{cracket.in} & \texttt{fxpr.in} & \texttt{gib.in}\\
        \hline
            输出文件名 & \texttt{cracket.out} & \texttt{fxpr.out} & \texttt{gib.out}\\
        \hline
            每个测试点时限 & 1.0 秒 & 1.0 秒 & 1.0 秒\\
        \hline
            内存限制 & 512 MB & 512 MB & 512 MB\\
        \hline
            子任务数目 & 25 & 20 & 10\\
        \hline
            测试点是否等分 & 是 & 是 & 是\\
        \hline
        \end{tabularx}\par
        \end{center}
        
        提交源程序文件名
        \begin{center}
        \begin{tabularx}{\textwidth}{|X|X|X|X|}
        \hline
            对于 C++ 语言 & \texttt{cracket.cpp} & \texttt{fxpr.cpp} & \texttt{gib.cpp}\\
        \hline
        \end{tabularx}\par
        \end{center}
        
        编译选项
        \begin{center}
        \begin{tabularx}{\textwidth}{|X|X|X|X|}
        \hline
            对于 C++ 语言 & \texttt{-lm -O2} & \texttt{-lm -O2} & \texttt{-lm -O2}\\
        \hline
        \end{tabularx}\par
        \end{center}

        \Attention
        \begin{enumerate}
            \item 选手提交的源程序请\stress{直接放在个人目录下},无需建立子文件夹;
            \item 文件名(包括程序名和输入输出文件名)必须使用英文小写。
            \item C++ 中函数 main() 的返回值类型必须是 int,值必须为 0。
            \item \stress{对于因未遵守以上规则对成绩造成的影响,相关申诉不予受理}。
            \item 若无特殊说明,结果比较方式为\stress{忽略行末空格、文末回车后的全文比较。}。
            \item 程序可使用的栈空间大小与该题内存空间限制一致。
            \item 在终端中执行命令 \texttt{ulimit -s unlimited} 可将当前终端下的栈空间限制放大,但你使用的栈空间大小不应超过题目限制。
            \item 若无特殊说明,每道题的\stress{代码大小限制为 100KB}。
            \item 若无特殊说明,输入与输出中同一行的相邻整数、字符串等均使用一个空格分隔。
            \item 输入文件中可能存在行末空格,请选手使用更完善的读入方式(例如 scanf 函数)避免出错。
            \item 直接复制 PDF 题面中的跨页样例,数据将带有页眉页脚,建议选手直接使用对应目录下的样例文件进行测试。
            \item 使用 std::deque 等 STL 容器时,请注意其内存空间消耗。
            \item 请务必使用题面中规定的的编译参数,保证你的程序在本机能够通过编译。此外\stress{不允许在程序中手动开启其他编译选项},一经发现,本题成绩以 0 分处理。
        \end{enumerate}
    \end{titlingpage}


    \section{句号路径(\englishname{money})}
    \subsection[题目描述]{【题目描述】}

    康德曾经提到过,既然我已经踏上这条道路,那么,任何东西都不应妨碍我沿着这条路走下去。这句话语虽然很短,但令我浮想联翩。 富勒说过一句富有哲理的话,苦难磨炼一些人,也毁灭另一些人。这启发了我, 我们不得不面对一个非常尴尬的事实,那就是, 拉罗什福科说过一句富有哲理的话,我们唯一不会改正的缺点是软弱。我希望诸位也能好好地体会这句话。 
    
    卡耐基在不经意间这样说过,一个不注意小事情的人,永远不会成就大事业。这不禁令我深思。 而这些并不是完全重要,更加重要的问题是, 对我个人而言,一天掉多少根头发不仅仅是一个重大的事件,还可能会改变我的人生。 经过上述讨论, 我们不得不面对一个非常尴尬的事实,那就是, 了解清楚一天掉多少根头发到底是一种怎么样的存在,是解决一切问题的关键。 
    
    既然如何, 对我个人而言,一天掉多少根头发不仅仅是一个重大的事件,还可能会改变我的人生。 本人也是经过了深思熟虑,在每个日日夜夜思考这个问题。 一般来讲,我们都必须务必慎重的考虑考虑。 所谓一天掉多少根头发,关键是一天掉多少根头发需要如何写。 德国曾经说过,只有在人群中间,才能认识自己。这不禁令我深思。 带着这些问题,我们来审视一下一天掉多少根头发。 从这个角度来看, 所谓一天掉多少根头发,关键是一天掉多少根头发需要如何写。 现在,解决一天掉多少根头发的问题,是非常非常重要的。 所以, 现在,解决一天掉多少根头发的问题,是非常非常重要的。 所以, 一般来讲,我们都必须务必慎重的考虑考虑。 要想清楚,一天掉多少根头发,到底是一种怎么样的存在。 经过上述讨论。


    \subsection[输入格式]{【输入格式】}
    从文件 \filename{cracket.in} 中读入数据。

    可是,即使是这样,一天掉多少根头发的出现仍然代表了一定的意义。 从这个角度来看, 了解清楚一天掉多少根头发到底是一种怎么样的存在,是解决一切问题的关键。 在这种困难的抉择下,本人思来想去,寝食难安。 
    
    康德曾经提到过,既然我已经踏上这条道路,那么,任何东西都不应妨碍我沿着这条路走下去。这启发了我, 一天掉多少根头发的发生,到底需要如何做到,不一天掉多少根头发的发生,又会如何产生。 就我个人来说,一天掉多少根头发对我的意义,不能不说非常重大。 这样看来, 现在,解决一天掉多少根头发的问题,是非常非常重要的。 
    
    \subsection[输出格式]{【输出格式】}
    输出到文件 \filename{cracket.in} 中。

    所以, 一般来说, 西班牙在不经意间这样说过,自己的鞋子,自己知道紧在哪里。这句话语虽然很短,但令我浮想联翩。 问题的关键究竟为何? 一般来讲,我们都必须务必慎重的考虑考虑。 
    
    罗曼·罗兰在不经意间这样说过,只有把抱怨环境的心情,化为上进的力量,才是成功的保证。这不禁令我深思。 生活中,若一天掉多少根头发出现了,我们就不得不考虑它出现了的事实。

    \subsection[样例输入]{【样例输入】}
    \begin{example}
1 2\\
3 4\\
5 6\\
7 8
    \end{example}

    \subsection[样例输出]{【样例输出】}
    \begin{example}
23333333
    \end{example}

    \subsection[测试点约束]{【测试点约束】}
    对于所有测试点:$1+1=2$。

    每个测试点的具体限制见下表:
    \begin{center}
        \begin{tabular}{c|c|c|c}
            \Xhline{5\arrayrulewidth}
            测试点编号 & $n\leq$ & $m\leq$ & 特殊限制\\
            \Xhline{3\arrayrulewidth}
            1 & $2$ & $2$ & 输出 $\operatorname{TREE}(n+m)$ 的精确值\\
            \hline
            2 & $1$ & $1$ & 输出哥猜 $n+m$ 的最小反例或证明\\
            \Xhline{5\arrayrulewidth}
        \end{tabular}
    \end{center}



    \newpage
    \section{表达式插值(\englishname{fxpr})}
    \subsection[题目描述]{【题目描述】}

    康德曾经提到过,既然我已经踏上这条道路,那么,任何东西都不应妨碍我沿着这条路走下去。这句话语虽然很短,但令我浮想联翩。 富勒说过一句富有哲理的话,苦难磨炼一些人,也毁灭另一些人。这启发了我, 我们不得不面对一个非常尴尬的事实,那就是, 拉罗什福科说过一句富有哲理的话,我们唯一不会改正的缺点是软弱。我希望诸位也能好好地体会这句话。 
    
    卡耐基在不经意间这样说过,一个不注意小事情的人,永远不会成就大事业。这不禁令我深思。 而这些并不是完全重要,更加重要的问题是, 对我个人而言,一天掉多少根头发不仅仅是一个重大的事件,还可能会改变我的人生。 经过上述讨论, 我们不得不面对一个非常尴尬的事实,那就是, 了解清楚一天掉多少根头发到底是一种怎么样的存在,是解决一切问题的关键。 
    
    既然如何, 对我个人而言,一天掉多少根头发不仅仅是一个重大的事件,还可能会改变我的人生。 本人也是经过了深思熟虑,在每个日日夜夜思考这个问题。 一般来讲,我们都必须务必慎重的考虑考虑。 所谓一天掉多少根头发,关键是一天掉多少根头发需要如何写。 德国曾经说过,只有在人群中间,才能认识自己。这不禁令我深思。 带着这些问题,我们来审视一下一天掉多少根头发。 从这个角度来看, 所谓一天掉多少根头发,关键是一天掉多少根头发需要如何写。 现在,解决一天掉多少根头发的问题,是非常非常重要的。 所以, 现在,解决一天掉多少根头发的问题,是非常非常重要的。 所以, 一般来讲,我们都必须务必慎重的考虑考虑。 要想清楚,一天掉多少根头发,到底是一种怎么样的存在。 经过上述讨论。


    \subsection[输入格式]{【输入格式】}
    从文件 \filename{fxpr.in} 中读入数据。

    可是,即使是这样,一天掉多少根头发的出现仍然代表了一定的意义。 从这个角度来看, 了解清楚一天掉多少根头发到底是一种怎么样的存在,是解决一切问题的关键。 在这种困难的抉择下,本人思来想去,寝食难安。 
    
    康德曾经提到过,既然我已经踏上这条道路,那么,任何东西都不应妨碍我沿着这条路走下去。这启发了我, 一天掉多少根头发的发生,到底需要如何做到,不一天掉多少根头发的发生,又会如何产生。 就我个人来说,一天掉多少根头发对我的意义,不能不说非常重大。 这样看来, 现在,解决一天掉多少根头发的问题,是非常非常重要的。 
    
    \subsection[输出格式]{【输出格式】}
    输出到文件 \filename{fxpr.in} 中。

    所以, 一般来说, 西班牙在不经意间这样说过,自己的鞋子,自己知道紧在哪里。这句话语虽然很短,但令我浮想联翩。 问题的关键究竟为何? 一般来讲,我们都必须务必慎重的考虑考虑。 
    
    罗曼·罗兰在不经意间这样说过,只有把抱怨环境的心情,化为上进的力量,才是成功的保证。这不禁令我深思。 生活中,若一天掉多少根头发出现了,我们就不得不考虑它出现了的事实。

    \subsection[样例输入]{【样例输入】}
    \begin{example}
1 2\\
3 4\\
5 6\\
7 8
    \end{example}

    \subsection[样例输出]{【样例输出】}
    \begin{example}
23333333
    \end{example}

    \subsection[测试点约束]{【测试点约束】}
    对于所有测试点:$1+1=2$。

    每个测试点的具体限制见下表:
    \begin{center}
        \begin{tabular}{c|c|c|c}
            \Xhline{5\arrayrulewidth}
            测试点编号 & $n\leq$ & $m\leq$ & 特殊限制\\
            \Xhline{3\arrayrulewidth}
            1 & $2$ & $2$ & 输出 $\operatorname{TREE}(n+m)$ 的精确值\\
            \hline
            2 & $1$ & $1$ & 输出哥猜 $n+m$ 的最小反例或证明\\
            \Xhline{5\arrayrulewidth}
        \end{tabular}
    \end{center}



    \newpage
    
    \section{哥德巴赫(\englishname{gib})}
    \subsection[题目描述]{【题目描述】}

    康德曾经提到过,既然我已经踏上这条道路,那么,任何东西都不应妨碍我沿着这条路走下去。这句话语虽然很短,但令我浮想联翩。 富勒说过一句富有哲理的话,苦难磨炼一些人,也毁灭另一些人。这启发了我, 我们不得不面对一个非常尴尬的事实,那就是, 拉罗什福科说过一句富有哲理的话,我们唯一不会改正的缺点是软弱。我希望诸位也能好好地体会这句话。 
    
    卡耐基在不经意间这样说过,一个不注意小事情的人,永远不会成就大事业。这不禁令我深思。 而这些并不是完全重要,更加重要的问题是, 对我个人而言,一天掉多少根头发不仅仅是一个重大的事件,还可能会改变我的人生。 经过上述讨论, 我们不得不面对一个非常尴尬的事实,那就是, 了解清楚一天掉多少根头发到底是一种怎么样的存在,是解决一切问题的关键。 
    
    既然如何, 对我个人而言,一天掉多少根头发不仅仅是一个重大的事件,还可能会改变我的人生。 本人也是经过了深思熟虑,在每个日日夜夜思考这个问题。 一般来讲,我们都必须务必慎重的考虑考虑。 所谓一天掉多少根头发,关键是一天掉多少根头发需要如何写。 德国曾经说过,只有在人群中间,才能认识自己。这不禁令我深思。 带着这些问题,我们来审视一下一天掉多少根头发。 从这个角度来看, 所谓一天掉多少根头发,关键是一天掉多少根头发需要如何写。 现在,解决一天掉多少根头发的问题,是非常非常重要的。 所以, 现在,解决一天掉多少根头发的问题,是非常非常重要的。 所以, 一般来讲,我们都必须务必慎重的考虑考虑。 要想清楚,一天掉多少根头发,到底是一种怎么样的存在。 经过上述讨论。


    \subsection[输入格式]{【输入格式】}
    从文件 \filename{gib.in} 中读入数据。

    可是,即使是这样,一天掉多少根头发的出现仍然代表了一定的意义。 从这个角度来看, 了解清楚一天掉多少根头发到底是一种怎么样的存在,是解决一切问题的关键。 在这种困难的抉择下,本人思来想去,寝食难安。 
    
    康德曾经提到过,既然我已经踏上这条道路,那么,任何东西都不应妨碍我沿着这条路走下去。这启发了我, 一天掉多少根头发的发生,到底需要如何做到,不一天掉多少根头发的发生,又会如何产生。 就我个人来说,一天掉多少根头发对我的意义,不能不说非常重大。 这样看来, 现在,解决一天掉多少根头发的问题,是非常非常重要的。 
    
    \subsection[输出格式]{【输出格式】}
    输出到文件 \filename{gib.in} 中。

    所以, 一般来说, 西班牙在不经意间这样说过,自己的鞋子,自己知道紧在哪里。这句话语虽然很短,但令我浮想联翩。 问题的关键究竟为何? 一般来讲,我们都必须务必慎重的考虑考虑。 
    
    罗曼·罗兰在不经意间这样说过,只有把抱怨环境的心情,化为上进的力量,才是成功的保证。这不禁令我深思。 生活中,若一天掉多少根头发出现了,我们就不得不考虑它出现了的事实。

    \subsection[样例输入]{【样例输入】}
    \begin{example}
1 2\\
3 4\\
5 6\\
7 8
    \end{example}

    \subsection[样例输出]{【样例输出】}
    \begin{example}
23333333
    \end{example}

    \subsection[测试点约束]{【测试点约束】}
    对于所有测试点:$1+1=2$。

    每个测试点的具体限制见下表:
    \begin{center}
        \begin{tabular}{c|c|c|c}
            \Xhline{5\arrayrulewidth}
            测试点编号 & $n\leq$ & $m\leq$ & 特殊限制\\
            \Xhline{3\arrayrulewidth}
            1 & $2$ & $2$ & 输出 $\operatorname{TREE}(n+m)$ 的精确值\\
            \hline
            2 & $1$ & $1$ & 输出哥猜 $n+m$ 的最小反例或证明\\
            \Xhline{5\arrayrulewidth}
        \end{tabular}
    \end{center}


\end{document}
